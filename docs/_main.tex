% Options for packages loaded elsewhere
\PassOptionsToPackage{unicode}{hyperref}
\PassOptionsToPackage{hyphens}{url}
%
\documentclass[
]{book}
\title{SPARTYN Supplementary material}
\author{Nate Osher}
\date{2021-11-27}

\usepackage{amsmath,amssymb}
\usepackage{lmodern}
\usepackage{iftex}
\ifPDFTeX
  \usepackage[T1]{fontenc}
  \usepackage[utf8]{inputenc}
  \usepackage{textcomp} % provide euro and other symbols
\else % if luatex or xetex
  \usepackage{unicode-math}
  \defaultfontfeatures{Scale=MatchLowercase}
  \defaultfontfeatures[\rmfamily]{Ligatures=TeX,Scale=1}
\fi
% Use upquote if available, for straight quotes in verbatim environments
\IfFileExists{upquote.sty}{\usepackage{upquote}}{}
\IfFileExists{microtype.sty}{% use microtype if available
  \usepackage[]{microtype}
  \UseMicrotypeSet[protrusion]{basicmath} % disable protrusion for tt fonts
}{}
\makeatletter
\@ifundefined{KOMAClassName}{% if non-KOMA class
  \IfFileExists{parskip.sty}{%
    \usepackage{parskip}
  }{% else
    \setlength{\parindent}{0pt}
    \setlength{\parskip}{6pt plus 2pt minus 1pt}}
}{% if KOMA class
  \KOMAoptions{parskip=half}}
\makeatother
\usepackage{xcolor}
\IfFileExists{xurl.sty}{\usepackage{xurl}}{} % add URL line breaks if available
\IfFileExists{bookmark.sty}{\usepackage{bookmark}}{\usepackage{hyperref}}
\hypersetup{
  pdftitle={SPARTYN Supplementary material},
  pdfauthor={Nate Osher},
  hidelinks,
  pdfcreator={LaTeX via pandoc}}
\urlstyle{same} % disable monospaced font for URLs
\usepackage{longtable,booktabs,array}
\usepackage{calc} % for calculating minipage widths
% Correct order of tables after \paragraph or \subparagraph
\usepackage{etoolbox}
\makeatletter
\patchcmd\longtable{\par}{\if@noskipsec\mbox{}\fi\par}{}{}
\makeatother
% Allow footnotes in longtable head/foot
\IfFileExists{footnotehyper.sty}{\usepackage{footnotehyper}}{\usepackage{footnote}}
\makesavenoteenv{longtable}
\usepackage{graphicx}
\makeatletter
\def\maxwidth{\ifdim\Gin@nat@width>\linewidth\linewidth\else\Gin@nat@width\fi}
\def\maxheight{\ifdim\Gin@nat@height>\textheight\textheight\else\Gin@nat@height\fi}
\makeatother
% Scale images if necessary, so that they will not overflow the page
% margins by default, and it is still possible to overwrite the defaults
% using explicit options in \includegraphics[width, height, ...]{}
\setkeys{Gin}{width=\maxwidth,height=\maxheight,keepaspectratio}
% Set default figure placement to htbp
\makeatletter
\def\fps@figure{htbp}
\makeatother
\setlength{\emergencystretch}{3em} % prevent overfull lines
\providecommand{\tightlist}{%
  \setlength{\itemsep}{0pt}\setlength{\parskip}{0pt}}
\setcounter{secnumdepth}{5}
\usepackage{booktabs}
\usepackage{bm}
\ifLuaTeX
  \usepackage{selnolig}  % disable illegal ligatures
\fi
\usepackage[]{natbib}
\bibliographystyle{plainnat}

\begin{document}
\maketitle

{
\setcounter{tocdepth}{1}
\tableofcontents
}
\hypertarget{about}{%
\chapter{About}\label{about}}

The purpose of this document is to serve as a ``living'' supplement to the first paper on the SPARTYN pipeline. While the paper itself will be static upon publication, this document will serve as both an ongoing supplement as well as a central location to give updated, detailed information on the pipeline and methods.

If you have a question that is not adequately addressed in this supplement, please email oshern (at) umich.edu.

\hypertarget{introduction}{%
\chapter{Introduction}\label{introduction}}

While the staining and examination of tissue samples has been a
ubiquitous medical practice for decades at this point in time, it
has only been relatively recently that high definition images of
such stainings have begun to be explored through the lens of
machine learning and statistical modeling. A considerable amount
of work has thus far been put into building and training models
capable of remarkably accurate classifications of overall tissue
samples as well as subsections of tissue samples {[}Saltz et al 2018;
TODO: other examples{]}. While valuable in their own right such
advances have also created additional opportunities for the
application of more traditional statistical techniques.
Notably, the ability to quickly and accurately process high definition
images of biopsies into cell level classification and location data
has allowed for the development and application of more traditional
spatial statistical modeling methods.

There is ample prior reason to suspect that such applications may
be valuable, particularly in the domain of cancer pathology. The
longstanding conventional wisdom among pathologists is that
tissue features such as lymphocyte infiltration as well as general
tissue heterogeneity are meaningful prognostic indicators {[}TODO:
Citations{]}. So far, this conventional wisdom has found support in
the current quantitative histopathological imaging analysis literature.
Li et al.~(2018) found that the spatial associations between stromal
cells and other types of cells were significantly associated with
survival in non-small cell lung cancer using both a Hidden Potts
Mixture Model as well as a mark interaction model {[}Note: these are
two different papers, both by Li in 2018- not sure the correct way
to acknowledge{]}. Saltz et al.~(2018) specifically examined the
presence of lymphocytes across biopsies in several types of cancer, and
found that certain summarization metrics of lymphocyte clusters were
significantly associated with survival in certain types of cancer.

In this paper, we introduce the SPARTYN (SPatial Analysis of
paRtitioned Tumor-lYmphocyte imagiNg) pipeline. SPARTYN is unique
in the spatial pathological imaging analysis literature in that it
uses statistical models to assess the association between tumor
cells and lymphocytes across an entire partitioned biopsy. This
allows for rigorous quantification of uncertainty in the style of Li
et al.~(2018) while still allowing for the assessment of full images
in the style of Saltz et al.~(2018). We accomplish this by
partitioning the cell-level imaging data of each biopsy into non-
overlapping sub-regions, which can then be modeled to capture the
local infiltration patterns using standard techniques from
point process theory.

Spatial point processes have long been used in the domain of
ecology to rigorously investigate spatial relationships between
various organisms (H\{"o\}gmander and S\{"a\}rkk\{"a\}, 1999; King et
al.~2012). More recently, methods from this paradigm have been
successfully applied within the domain of Biostatistics (Kang
et al.~2011; Kang et al.~2015). Given the relatively simple and
granular nature of cell-level spatial imaging data, marked point
processes are a natural way to simultaneously model the randomness
in both the quantities and locations of the different cell types
along with their relative spatial associations.

\hypertarget{materials-and-methods}{%
\chapter{Materials and Methods}\label{materials-and-methods}}

\hypertarget{overview}{%
\section{Overview}\label{overview}}

Raw data was obtained for this project using the image
processing tools of Rao and Krishnan. A random forest model was
trained on high definition images of cancer biopsies in order
to automatically classify cells of different types. Images of
Skin Cutaneous Melanoma (SKCM) biopsies from The Cancer Genome Atlas
program were processed using this model such that
each cell was classified as a tumor cell, a lymphocyte, or
other. In addition, the x- and y-coordinates of each cell
centroid (relative to the pathology slide) were determined and
recorded. The resulting data set for each biopsy consisted of a
row for each cell, with a column for the x-coordinate, a column
for the y-coordinate, and an indicator of the cell type. Each
biopsy was intensity thresholded to define a ``fitted'' window, i.e.
the smallest window that fit all cells in the biopsy within it.
This window was then divided into tiles that fully partition it,
while containing similar numbers of tumor cells. A bayesian spatial
point process model was then fit on each of these tiles separately,
yielding a posterior distribution of the local interaction parameter.
For each tile, this local distribution was then compared to a
tile-specific null distribution, before being combined to summarize
the overall level of interaction at the biopsy level.

SKCM is an appealing target for the investigation of lymphocyte
infiltration for several reasons. SKCM has been shown to be
particularly responsive to Immunotherapy in some cases {[}CITATION{]}.
It is possible that the ability to quantify lymphocyte
infiltration at a large scale may allow for more detailed
investigation into the scenarios in which this treatment may be
most effectively deployed. What's more, SKCM has an unusually
high mutational load amongst the various cancer types {[}CITATION{]}.
The ability to quantify infiltration may allow for further
investigation into not only genomic associations of this
occurrence but associations with mutations as well. {[}TODO:
is infiltration particularly common in SKCM? I feel like it is,
but I need to verify this.{]}

\hypertarget{spatial-point-processes}{%
\section{Spatial Point Processes}\label{spatial-point-processes}}

Denote the number of biopsies \(n\), and for biopsy \(i\), let \(c_i\)
denote the number of cells observed and labeled within that
biopsy, with corresponding x- and y- coordinates
\(\mathbf{x}_i = x_{i1},...,x_{ic_i}\), \(\mathbf{y}_i = y_{i1},...,y_{ic_i}\),
and marks
\(\mathbf{m}_i = m_{i1},...,m_{ic_i}\). For our application,
\(m_{ij} \in \{1, 2\}\),with \(1\) indicating a tumor cell and \(2\) indicating a
lymphocyte. Further, denote the number of tumor cells observed in
subject \(i\) by \(T_i\), and the number of lymphocytes \(L_i\).
Within each patient, this data can be naturally thought
of as a marked point process and modeled as such.

We ultimately decided to use a Hierachical
Multitype Strauss Model for our data, the density of which is

\[ f(\mathbf{p}_1, \mathbf{p}_2) \propto 
\beta_1^{n_1} \beta_2^{n_2}
\gamma_{11}^{S_{R_{11}}(\mathbf{p}_1)}
\gamma_{22}^{S_{R_{2}}(\mathbf{p}_2)}
\gamma_{12}^{S_{R_{12}}(\mathbf{p}_1, \mathbf{p}_2)}
\quad (1)\]

Where:

\begin{itemize}
\tightlist
\item
  \(\mathbf{p}_i\) is a vector of points of type \(i\)
\item
  \(n_i\) is the number of points of type \(i\)
\item
  \(\beta_i\) is the first order intensity of points of type \(i\)
\item
  \(S_{R_{ij}}(\cdot)\) counts the number of pairs of points of types \(i\) and \(j\) within \(R_{ij}\) one another, where \(R_{ij}\) is selected a priori based on subject specific knowledge.
\item
  \(\gamma_{ij}\) captures the tendency of points of type \(i\) to be near points of type \(j\)
\end{itemize}

We decided to use this model over the standard Multitype Strauss
model for two reasons.
Firstly, treating the locations of the lymphocytes as conditional
upon the locations of the tumor cells is a priori biologically
plausible. Secondly, the hierarchical variant of the Strauss
model allows for the proper modeling of positive interaction
between points of different types, while the standard
multitype model does not. Note that because intra-type
interaction is still confined to be negative under the
Hierarchical Multitype Strauss Process, we constrained the
intra-type interaction parameters \(\gamma_{11}\) and \(\gamma_{22}\)
to be 1. Finally, based on prior biological knowledge, we set
\(R_{12} = 30\) \(\mu m\). In our context, \(\gamma_{12}\) can be
thought of as the degree to which lymphocytes to be close to or
far away from tumor cells, conditional upon the locations
of the tumor cells. This allows us to distinguish between the\\
mere relative abundance of different cell types (which is\\
captured by the \(\beta_1\) and \(\beta_2\) parameters) and the
actual spatial associations between the different cell types.

\hypertarget{intensity-thresholding-and-partition}{%
\section{Intensity Thresholding and Partition}\label{intensity-thresholding-and-partition}}

Each biopsy was partitioned into non-overlapping sub-regions,
each of which was modeled as a point process. The primary
motivation for structuring the analysis this came from prior
biological knowledge about tumor composition. Tumors are
heterogeneous entities in many respects, and lymphocyte infiltration
is no exception. Thus, partitioning the biopsy into non-
overlapping sub-regions and fitting models on each sub-region is
a natural way to capture this heterogeneity. This strategy has additional benefits in that it allows for the parallelization of
model fitting across the resulting sub-regions within a single
biopsy.

In order to partition a given biopsy into non-overlapping
sub-regions, we beganwith the smallest bounding rectangular
window that contained cells \(1...c_i\), we applied an intensity thresholding algorithm {[}too much?{]} in order to find the smallest
possible window that still contained all \(c_i\) cells. Next, we
applied a voronoi tesselation to the \(T_i\) tumor cells within the intensity thresholded window, partitioning it into tumor cell\\
specific sub-windows \(1...T_i\) corresponding to tumor cells\\
\(1...T_i\). We then applied a modified version of k-means to\\
tumor cells \(1...T_i\),such that each of the \(K_i\) resulting\\
clusters was constrained to be between a pre-defined range of\\
cell counts. Finally, each tumor cell specific sub-window within
a given cluster was combined into a tile, corresponding to each
of the \(K_i\) clusters from the k-means clustering. This results
in \(K_i\) tiles that fully partition the intensity thresholded
window. This partition uniquely defines the membership of each of
the \(c_i\) total cells into one of the resulting \(K_i\) total tiles.

It is worth emphasizing that there is nothing particularly unique
about this method of partitioning the biopsy. Any other method could
be used in its place, so long as the result is a partition of the
biopsy into some number of non-overlapping sub-regions on which the
subsequent model fitting can proceed.

\hypertarget{inference}{%
\section{Inference}\label{inference}}

In order to compute posterior distributions of parameters of
interest, we used Bayesian techniques in the style of King et al.
2012. This methodology essentially exchanges the likelihood
function used in standard Bayesian inference for the
pseudolikelihood function (Besag, 1975 {[}VERIFY{]}), with the
integral approximated via the Berman-Turner device (Baddeley
and Turner, 2000) using the spatstat package (Baddeley and Turner,
2005). Finally, each parameter estimated (\(\beta_1\), \(\beta_2\),
\(\gamma_{12}\)) was assigned a flat prior. {[}TODO: Technically,
normal with mean 0 and variance \(1,000,000\) because of limitations
of JAGS- should I spell this out?{]} Posterior distributions were
computed using MCMC via the R2jags package (Su, 2015). Using
these techniques we were able to compute a
posterior distribution \(f_{ik | \mathbf{p}}(y)\) of \(\gamma_{ik,12}\) for
each for each biopsy \(i \in \{1...n\}\) and tile
\(k \in \{1...K_i\}\).

\hypertarget{infiltration-probability}{%
\section{Infiltration Probability}\label{infiltration-probability}}

In order to compute a localized probability of infiltration, each
posterior distribution \(f_{ik | \mathbf{p}}(y)\) was compared to a
tile specific null distribution, \(f_{ik, 0}(y)\).
Because under null interaction (\(\gamma_{ik, 12} = 1)\)) our
model reduces to two independent Poisson processes with
intensities \(\beta_1, \beta_2\), this was accomplished by
treating the observed tumor cells as fixed, simulating \(s\)
realizations of lymphocytes under a Poisson process with the
observed intensity, performing the previously described model
fitting procedure on each, and aggregating samples across
simulation \(1...s\). Under the assumption that the posterior
distribution \(f_{ik | \mathbf{p}}(x)\) is independent from the null
distribution \(f_{ik, 0}(y)\), we then computed the tile specific
Infiltration Probability \(r_{ik}\), given by

\[\text{Infiltration Probability} \quad r_{ik} = \int \int  I(x > y) f_{ik | \mathbf{p}}(x) f_{ik, 0}(y)  dxdy \qquad (2)\]

Note that \(r_{ik} \in [0, 1]\). Because \(r_{ik}\) is computed as
the integral over an indicator variable, this naturally yields
an interpretation as a probability. Specifically, \(r_{ik}\) can
be thought of as measuring the posterior probability that the
observed value of \(\gamma_{12}\) (denoted by \(x\) in the
integral) is larger than the value of \(\gamma_{12}\) we would
expect to observe by chance (denoted by \(y\) in the integral).
We thus refer to \(r_{ik}\) as the Infiltration Probability
for that particular tile, which can be aggregated across tiles
to yield a measure of infiltration on a particular biopsy.

\hypertarget{results}{%
\chapter{Results}\label{results}}

\hypertarget{simulation}{%
\section{Simulation}\label{simulation}}

To compare our model's detection of spatial association between
different cell types, we ran a small-scale simulation study in
which we tested the ability of this method to accurately classify
positive interaction across a range of different simulated cell
compositions and spatial associations. For each of four sets of
simulations, the number of simulated tumor cells and lymphocytes
were set a priori at \(T_s\) and \(L_s\) respectively. Further,
interaction was controlled by a parameter \(\phi \in [-1, 1]\), with
\(-1\) indicating the most negative possible interaction and \(1\)
indicating the most positive possible interaction.

For a given combination of \(T_s\) and \(L_s\), the positive and
negative simulations proceeded differently. For the negative
simulations (\(\phi \in [-1, 0]\)), \(T_s\) tumor cells and \(L_s\)
lymphocytes were simulated as independent poisson processes in
regions of the window that overlapped to varying degrees. The
overlap was controlled by \(\phi\), such that the processes overlapped
on \((100 \cdot (1 + \phi)) \%\) of the window in which the
simulation occurred. Note that when \(\phi = -1\) there was no overlap,
and when \(\phi = 0\) (complete overlap) the simulation reduced to
simulating two independent Poisson Processes within the same window.

For the positive simulations (\(\phi \in (0, 1]\)), \(T_s\) tumor cells
were simulated under a Poisson process. After their locations were
determined, \(L_s\) lymphocytes were simulated. For each lymphocyte
\(l_i\), a Bernoulli random variable \(C_i \thicksim Bern(\phi)\) was
drawn. If \(C_i = 1\), \(l_i\) was simulated within \(30\) \(\mu m\) of a\\
randomly selected tumor cell \(t_j\). Otherwise, \(l_i\) was simulated
from a Poisson process. Thus, the level of interaction was again
controlled by \(p\), with \(p = 0\) now corresponding to two independent
Poisson Processes and \(p = 1\) corresponding to a situation in which
all lymphocytes are within \(30\) \(\mu m\) of at least one tumor cell.

Across the different simulation settings, accurate classification
was possible using IP, with the minimum AUC across simulations
being \(0.84\). For detailed descriptions of simulation settings,
simulated data, and results, see Figure {[}N{]}.

\hypertarget{application}{%
\section{Application}\label{application}}

We applied our method to a data set consisting of 335 images of
skin cutaneous melanoma taken from The Cancer Genome Atlas. Images
were processed as described in section 2.1. Our methods were applied
to the resulting data sets. {[}GIVE SUMMARY STATISTICS ABOUT TILES?{]}

\hypertarget{survival-analysis}{%
\subsection{Survival Analysis}\label{survival-analysis}}

In order to assess association between PPPI and survival, we fit
a Cox Proportional Hazards model. In addition to adjusting for
average logit-PPPI, we adjusted for patient level factors such as
cancer stage, age, and sex. In addition, we adjusted for readily
calculable tumor level features, such as number of tumor cells (as
a proxy for size) and logit lymphocyte proportion (the number of
TILs divided by the number of tumor cells and TILs). We found that
after adjusting for these other factors, an increase in logit-PPPI
was significantly associated with increased risk of death
(\(p < 0.05\)). The same model was fitted with average logit-PPPI
exchanged for the average normalized value of the estimated Mark
Correlation Function evaluated at \(r = 30\) (the same as the radius
of interaction used in our model fitting). See Table 1 for
coefficients and standard errors in each model.

\hypertarget{genomic-associations}{%
\subsection{Genomic Associations}\label{genomic-associations}}

In addition to associations with survival, we investigated the
association between our measurement and gene expression. Gene
expression {[}RNA-seq{]} data was acquired for all 335 patients in our
sample using TCGA Assembler {[}citation here{]}. Additionally, 42
significantly mutated genes of interest were identified using
previous work investigating the genomic differences in SCM {[}TCGA
Network, 2015{]}. Of the 335 patients in our sample, 240 had gene
expression data for all genes of interest, while 95 were missing
data for all genes of interest. We examined the marginal association
between the normalized gene expression data for the 240 patients with
complete data and average logit PPPI values. After correcting for
multiple testing using the Benjamini-Hochberg procedure, we found that
the expression of three genes were significantly associated with
average logit PPPI: LRRC37A3, B2M, and TP53. See supplementary Table
{[}N{]} for full details on significance of associations with each gene.

\hypertarget{conclusion}{%
\chapter{Conclusion}\label{conclusion}}

As algorithms for cell-level image classification improve, the
opportunities for more and more granular quantitative analysis of
histopathological imaging data will become both more numerous and
more fruitful. Moreover, as spatial gene expression data becomes
more and more common, so too will opportunities for synthesizing
data on the relative spatial locations of different cell types
along with local gene expression data through complex modeling.

The SPARTYN pipeline represents a valuable contribution in and of
itself to the histopathological imaging analysis literature through
its ability to model and quantify lymphocyte infiltration across
entire biopsies in a way that captures meaningful variation across
patients. However, it also creates numerous opportunities for future
work. Partitioning each biopsy into non-overlapping sub-regions such
that each is assigned a value invites the usage of other tools from
the spatial statistical canon. Specifically, tools from areal data
analysis may be readily applied without any modifications or further
theoretical development. Moreover, recall that our usage of Bayesian
methods allows for not only the computation of scalar summary
statistics for each sub-region but also posterior distributions.
Future work may involve adapting more traditional methods to model
correlated density functions, or developing new methods as necessary.

It also bears mentioning that while we limited ourselves to investigating
two cell types throughout this paper, there is nothing in our pipeline
or model that depends on this limitation in such a way that would render
future investigation of more cell types impossible. It is highly likely
that future iterations of our pipeline and methodology will seek to
investigate the spatial associations between three or more cell types.

Finally, our work could very easily take advantage of the burgeoning
field of spatial transcriptomics. It would be quite trivial
(particularly in a Bayesian framework) to model a function of one or
all of the parameters of interest as a linear combination of the
local gene expression data for genes that are known to be relevant
to immune response. Alternatively, future methods in this area could
be developed at the intersection of spatial data and high dimensional\\
data in order to identify such genes.

  \bibliography{book.bib,packages.bib}

\end{document}
